What is Heritability?
Heritability refers to the proportion of phenotypic variance in a population that can be attributed to genetic variance. It is an important concept in genetics and evolutionary biology, as it helps to understand how traits are passed from one generation to the next.
Types of Heritability
Narrow-Sense Heritability (h²):
Represents the proportion of phenotypic variance that is due to additive genetic variance (the sum of the effects of individual alleles).
Calculated as: h2=VAVPh^2 = \frac{V_A}{V_P}h2=VP​VA​​
Where VAV_AVA​ is additive genetic variance, and VPV_PVP​ is total phenotypic variance.
Broad-Sense Heritability (H²):
Includes all genetic variance (additive, dominance, and epistatic variance).
Calculated as: H2=VGVPH^2 = \frac{V_G}{V_P}H2=VP​VG​​
Where VGV_GVG​ is total genetic variance.
Phenotypic Variance
Phenotypic variance (VPV_PVP​) can be broken down into several components:
Genetic Variance (V_G): Variance due to differences in genotype.
Environmental Variance (V_E): Variance due to environmental factors.
Interaction Variance (V_GE): Variance due to interactions between genetic and environmental factors.
Thus, the total phenotypic variance can be expressed as:
VP=VG+VEV_P = V_G + V_EVP​=VG​+VE​
Breeder's Equation
The Breeder's Equation is a fundamental equation in quantitative genetics that relates the response to selection to heritability and selection differential. It is expressed as:
R=h2SR = h^2 SR=h2S
R: Response to selection (the change in the trait mean in the next generation).
h²: Narrow-sense heritability.
S: Selection differential (the difference between the mean trait value of the selected individuals and the mean trait value of the entire population).
Understanding Breeder's Equation
Response to Selection (R):
Indicates how much a trait will change in response to selection.
A higher heritability (h²) implies a greater potential for change.
Selection Differential (S):
Reflects the strength of selection. A larger selection differential means stronger selection, leading to a larger change in the mean trait value.
Applications of Heritability
Animal and Plant Breeding: Understanding heritability helps breeders select for desirable traits.
Evolutionary Biology: Heritability informs us about how traits evolve in populations and how they might respond to environmental changes.
Behavioral Genetics: Helps in studying the genetic basis of behavior.
Limitations of Heritability
Population-Specific: Heritability estimates are specific to the population and environment from which they are derived. They may not apply to other populations.
Does Not Imply Determinism: A high heritability does not mean a trait is completely determined by genetics. Environmental factors still play a significant role.
Static Estimate: Heritability is a snapshot based on a specific population at a specific time and may change with environmental conditions or demographic shifts.
Key Takeaways
Heritability is crucial for understanding the genetic basis of traits and their evolution.
Narrow-sense heritability focuses on additive genetic effects, while broad-sense heritability includes all genetic influences.
The Breeder's Equation links heritability to the response of traits to selection, guiding breeding and evolutionary predictions.
Understanding the limitations of heritability is essential for interpreting genetic research and applying it in practical settings.
This overview should provide a solid foundation for understanding heritability and its implications in biology! If you have any specific areas you’d like to delve deeper into, just let me know!


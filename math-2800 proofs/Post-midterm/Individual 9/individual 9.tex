\documentclass[12pt]{article}
 
\usepackage[margin=1in]{geometry} 
\usepackage{amsmath,amsthm,amssymb}
\usepackage{comment}
 
\newenvironment{theorem}[2][Theorem]{\begin{trivlist}
\item[\hskip \labelsep {\bfseries #1}\hskip \labelsep {\bfseries #2.}]}{\end{trivlist}}
\newenvironment{lemma}[2][Lemma]{\begin{trivlist}
\item[\hskip \labelsep {\bfseries #1}\hskip \labelsep {\bfseries #2.}]}{\end{trivlist}}
\newenvironment{exercise}[2][Exercise]{\begin{trivlist}
\item[\hskip \labelsep {\bfseries #1}\hskip \labelsep {\bfseries #2.}]}{\end{trivlist}}
\newenvironment{problem}[2][Problem]{\begin{trivlist}
\item[\hskip \labelsep {\bfseries #1}\hskip \labelsep {\bfseries #2.}]}{\end{trivlist}}
\newenvironment{question}[2][Question]{\begin{trivlist}
\item[\hskip \labelsep {\bfseries #1}\hskip \labelsep {\bfseries #2.}]}{\end{trivlist}}
\newenvironment{corollary}[2][Corollary]{\begin{trivlist}
\item[\hskip \labelsep {\bfseries #1}\hskip \labelsep {\bfseries #2.}]}{\end{trivlist}}

% Solutions use a modified proof environment
\newenvironment{solution}
               {\let\oldqedsymbol=\qedsymbol
                \renewcommand{\qedsymbol}{$\blacktriangleleft$}
                \begin{proof}[\textit\upshape Solution]}
               {\end{proof}
                \renewcommand{\qedsymbol}{\oldqedsymbol}}
 
\begin{document}

% --------------------------------------------------------------
%                         Start here
% --------------------------------------------------------------

\title{Individual 9}%replace X with the appropriate number
\author{Yu Fan Mei\\
	Introduction to Proof and Problem Solving} %if necessary, replace with your course title

\maketitle

\begin{problem}{1}
    Show that  
\[
C((0, 1]) = C((0, 1))
\]  
by showing that the function  
\[
f(x) = 
\begin{cases} 
x & \text{if } x \neq \frac{1}{n} \text{ for any } n \in \mathbb{Z}^+ \\
\frac{1}{n+1} & \text{if } x = \frac{1}{n} \text{ for some } n \in \mathbb{Z}^+
\end{cases}
\]
is one-to-one from \((0, 1]\) onto \((0, 1)\). It might help to graph the function.
\end{problem}

\begin{proof} In order to prove these two sets have the same cardinality, we will prove that the function $f : (0, 1] \to (0, 1) $ is one-to-one and onto. Let $x_0$ be any real number such that $x_0 \in (0, 1]$

\end{proof}


\noindent While working on this proof, I received no external assistance aside from advice from Professor Mehmetaj.

\end{document}
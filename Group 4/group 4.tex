
% --------------------------------------------------------------
% This is all preamble stuff that you don't have to worry about.
% Head down to where it says "Start here"
% --------------------------------------------------------------

\documentclass[12pt]{article}
 
\usepackage[margin=1in]{geometry} 
\usepackage{amsmath,amsthm,amssymb}
\usepackage{comment}
 

 
\newenvironment{theorem}[2][Theorem]{\begin{trivlist}
\item[\hskip \labelsep {\bfseries #1}\hskip \labelsep {\bfseries #2.}]}{\end{trivlist}}
\newenvironment{lemma}[2][Lemma]{\begin{trivlist}
\item[\hskip \labelsep {\bfseries #1}\hskip \labelsep {\bfseries #2.}]}{\end{trivlist}}
\newenvironment{exercise}[2][Exercise]{\begin{trivlist}
\item[\hskip \labelsep {\bfseries #1}\hskip \labelsep {\bfseries #2.}]}{\end{trivlist}}
\newenvironment{problem}[2][Problem]{\begin{trivlist}
\item[\hskip \labelsep {\bfseries #1}\hskip \labelsep {\bfseries #2.}]}{\end{trivlist}}
\newenvironment{question}[2][Question]{\begin{trivlist}
\item[\hskip \labelsep {\bfseries #1}\hskip \labelsep {\bfseries #2.}]}{\end{trivlist}}
\newenvironment{corollary}[2][Corollary]{\begin{trivlist}
\item[\hskip \labelsep {\bfseries #1}\hskip \labelsep {\bfseries #2.}]}{\end{trivlist}}
\newenvironment{example}[2][Example]{\begin{trivlist}
\item[\hskip \labelsep {\bfseries #1}\hskip \labelsep {\bfseries #2.}]}{\end{trivlist}}

% Solutions use a modified proof environment
\newenvironment{solution}
               {\let\oldqedsymbol=\qedsymbol
                \renewcommand{\qedsymbol}{$\blacktriangleleft$}
                \begin{proof}[\textit\upshape Solution]}
               {\end{proof}
                \renewcommand{\qedsymbol}{\oldqedsymbol}}
 
\begin{document}

% --------------------------------------------------------------
%                         Start here
% --------------------------------------------------------------

\title{Group 4}%replace X with the appropriate number
\author{Helen Freedman, Pruthvi Jasty, Yu Fan Mei\\ %replace with your name
	Introduction to Proof and Problem Solving} %if necessary, replace with your course title
\maketitle

\begin{problem}{4a} %You can use theorem, exercise, problem, or question here.  Modify x.yz to be whatever number you are proving

    $$P_1 (f) \equiv \left\{ \forall M \in \mathbb{R}, \exists x \in \mathbb{R} \text{ such that } f(x) > M \right\}$$

    $$P_6 (f) \equiv \left\{ \exists (M, K) \in \mathbb{R}^2, \text{ such that } \forall x > K, f(x) > M \right\}$$

    Prove or disprove

    $$ \{ \forall f \text{ satisfying } P_6, f \text{ satisfies } P_1 \}. $$

\end{problem}

\begin{example}{1} It is fairly obvious that $P_1$ means the function is unbounded from above. To better understand the problem, we'll try to find functions that satisfy $P_6$.



    Let $f_0(x) = 2$. Set $M = 1$ and $K = 1$. Let $x$ be any real number larger than $K$. We can observe that $f_0(x) = 2 > M$ for all $x$ in this case.

    Let $f_1(x) = x$. Set $M = 0$ and $K = 1$. Let $x$ be any real number larger than $K$. We can see that $K > M$. We know that $f_1(x) = x > K$, so $f_1(x) > M$ holds for positive linear and other increasing functions.

    %Let $f_2(x) = -x$. Set $M = 0$ and $K = 1$. Let $x$ be any real number %larger than $K$. Since $x > K$, we can see that $f(x) < K$ since $x$



\end{example}

\begin{proof} We will disprove the statement. The negation of the statement is as follows:

    $$\{ \exists f \text{ that satisfies } P_6 \text{ and satisfies } \lnot P_1 \}$$

    The negation of $P_1$ is:

    $$\lnot P_1 \equiv \{ \exists M \in \mathbb{R} \text{ such that } \forall x \in \mathbb{R}, f(x) \leq M\}$$

    Set $f_0 = \sin(x)$. Set $(M_0, K_0) = (-2, 0)$. Let $x_0$ be any real number greater than $K$. We know that $-1 \leq \sin(n) \leq 1$ for any real number $n$, so $f_0(x_0) = \sin(x_0) \geq -1$.

    We know that $-1 > -2$, so $f_0(x) > -2 = M$. Thus, $f_0(x)$ satisfies $P_6.$

    Set $(M_0, K_0) = (2, 0)$. Let $x_0$ be any real number. Since $-1 \leq \sin(n) \leq 1$ for any real number $n$, so $f_0(x_0) = \sin(x_0) \leq 1$.

    We know that $2 > 1$, so $f_0(x) < 2 = M$. Thus, $f_0(x)$ satisfies $\lnot P_1$. 


\end{proof}


\noindent While working on this proof, we received no external assistance aside from advice from Professor Mehmetaj.

\end{document}

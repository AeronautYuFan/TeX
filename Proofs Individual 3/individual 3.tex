
% --------------------------------------------------------------
% This is all preamble stuff that you don't have to worry about.
% Head down to where it says "Start here"
% --------------------------------------------------------------

\documentclass[12pt]{article}
 
\usepackage[margin=1in]{geometry} 
\usepackage{amsmath,amsthm,amssymb}
 

 
\newenvironment{theorem}[2][Theorem]{\begin{trivlist}
\item[\hskip \labelsep {\bfseries #1}\hskip \labelsep {\bfseries #2.}]}{\end{trivlist}}
\newenvironment{lemma}[2][Lemma]{\begin{trivlist}
\item[\hskip \labelsep {\bfseries #1}\hskip \labelsep {\bfseries #2.}]}{\end{trivlist}}
\newenvironment{exercise}[2][Exercise]{\begin{trivlist}
\item[\hskip \labelsep {\bfseries #1}\hskip \labelsep {\bfseries #2.}]}{\end{trivlist}}
\newenvironment{problem}[2][Problem]{\begin{trivlist}
\item[\hskip \labelsep {\bfseries #1}\hskip \labelsep {\bfseries #2.}]}{\end{trivlist}}
\newenvironment{question}[2][Question]{\begin{trivlist}
\item[\hskip \labelsep {\bfseries #1}\hskip \labelsep {\bfseries #2.}]}{\end{trivlist}}
\newenvironment{corollary}[2][Corollary]{\begin{trivlist}
\item[\hskip \labelsep {\bfseries #1}\hskip \labelsep {\bfseries #2.}]}{\end{trivlist}}

% Solutions use a modified proof environment
\newenvironment{solution}
               {\let\oldqedsymbol=\qedsymbol
                \renewcommand{\qedsymbol}{$\blacktriangleleft$}
                \begin{proof}[\textit\upshape Solution]}
               {\end{proof}
                \renewcommand{\qedsymbol}{\oldqedsymbol}}
 
\begin{document}
 
% --------------------------------------------------------------
%                         Start here
% --------------------------------------------------------------
 
\title{Individual 3}%replace X with the appropriate number
\author{Yu Fan Mei\\ %replace with your name
Introduction to Proof and Problem Solving} %if necessary, replace with your course title
 
\maketitle
 
\begin{problem}{18} %You can use theorem, exercise, problem, or question here.  Modify x.yz to be whatever number you are proving
Show that the function $f$ mapping $S = \mathbb{R}$ into $T = \mathbb{R}$ is onto $T$, where

$$f(x) = \begin{cases} x+3 & x \not\in \mathbb{Z} \\
    x-2 & x \in \mathbb{Z} \end{cases} $$

\end{problem}
 
\begin{proof} Let $y_0$ be an arbitrary real number. We consider two cases. 

\noindent Case 1: Suppose $y_0$ is an integer. Set $x_0$ = $y_0 + 2$. Since integers are closed under addition, $x_0$ is an integer. Subtracting $2$ from both sides, we get

$$x_0 - 2 = y_0.$$

\noindent Case 2: Suppose $y_0$ is not an integer. We know integers are closed under addition. Thus, if $y_0$ is not an integer, $y_0 - 3$ cannot be an integer. Set $x_0 = y_0 - 3$. Adding 3 to both sides, we get 

$$x_0 + 3 = y_0.$$

Since $x_0 \in \mathbb{Z}$ in case 1 and $x_0 \in \mathbb{R} - \mathbb{Z}$ in case 2, $x_0$ can be any real number and thus $x \in S$. Thus, we have proven $f : S \to T$ is onto $T$.


\end{proof}

While working on this proof, I had no external assistance aside from advice from Professor Mehmetaj.
 
\end{document}

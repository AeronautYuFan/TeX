
% --------------------------------------------------------------
% This is all preamble stuff that you don't have to worry about.
% Head down to where it says "Start here"
% --------------------------------------------------------------

\documentclass[12pt]{article}
 
\usepackage[margin=1in]{geometry} 
\usepackage{amsmath,amsthm,amssymb}
\usepackage{comment}
 

 
\newenvironment{theorem}[2][Theorem]{\begin{trivlist}
\item[\hskip \labelsep {\bfseries #1}\hskip \labelsep {\bfseries #2.}]}{\end{trivlist}}
\newenvironment{lemma}[2][Lemma]{\begin{trivlist}
\item[\hskip \labelsep {\bfseries #1}\hskip \labelsep {\bfseries #2.}]}{\end{trivlist}}
\newenvironment{exercise}[2][Exercise]{\begin{trivlist}
\item[\hskip \labelsep {\bfseries #1}\hskip \labelsep {\bfseries #2.}]}{\end{trivlist}}
\newenvironment{problem}[2][Problem]{\begin{trivlist}
\item[\hskip \labelsep {\bfseries #1}\hskip \labelsep {\bfseries #2.}]}{\end{trivlist}}
\newenvironment{question}[2][Question]{\begin{trivlist}
\item[\hskip \labelsep {\bfseries #1}\hskip \labelsep {\bfseries #2.}]}{\end{trivlist}}
\newenvironment{corollary}[2][Corollary]{\begin{trivlist}
\item[\hskip \labelsep {\bfseries #1}\hskip \labelsep {\bfseries #2.}]}{\end{trivlist}}

% Solutions use a modified proof environment
\newenvironment{solution}
               {\let\oldqedsymbol=\qedsymbol
                \renewcommand{\qedsymbol}{$\blacktriangleleft$}
                \begin{proof}[\textit\upshape Solution]}
               {\end{proof}
                \renewcommand{\qedsymbol}{\oldqedsymbol}}
 
\begin{document}

% --------------------------------------------------------------
%                         Start here
% --------------------------------------------------------------

\title{Individual 7}%replace X with the appropriate number
\author{Yu Fan Mei\\ %replace with your name
	Introduction to Proof and Problem Solving} %if necessary, replace with your course title

\maketitle

\begin{lemma}{1} For every integer $n > 0$, $n(n+1)$ is even. We will prove this via induction. \noindent \\ \\
    Base Case: Set $n = n_0 = 1$. Then $n_0(n_0+1) = 2$, which is clearly true. \\ \\ \noindent
    Inductive Step: Assume there exists integers $n_0 > 0$ and $k_0$ such that

    $$n_0(n_0+1) = 2k_0.$$

    We want to show that there exists an integer $m_0$ such that
    $$(n_0+1)(n_0+2) = 2m_0.$$
    Set $m_0 = (k_0 + n_0 + 1)$. Since $k_0$ is an integer and integers are closed under addition, $j_0$ is also an integer. Unfactoring the left side of the statement above, we get $(n_0+1)(n_0+2) = n_0^2 + 3n_0 + 2$. When we substitute $2k+1$ into this, we're left with $2k_0 + 2n_0 + 2$. When we rewrite this expression with the 2 factored, the equation becomes

    $$2(k_0 + n_0 + 1).$$

    We can rewrite this as $2m_0$, which is what we needed to show.
\end{lemma}

\begin{problem}{1}
    Use induction to prove that $(n^3 - n)(n + 2)$ is divisible by $12$ for all $n \geq 1$.
\end{problem}

\begin{proof} We will prove problem 1 using induction. \\ \\
    Base Case: Set $n = n_0 = 1$. Then
    $$(n^3 - n)(n + 2) = 0.$$

    This is clearly true, since 0 can be written as $12(0)$, and is thus divisible by 12. \\ \\

    Inductive Step: Assume there exists integers $n \geq 1$ and $k_0$ such that

    $$(n^3 - n)(n + 2) = 12k_0.$$

    We want to show that this is true for $n + 1$, meaning we want to show that there exists an integer $j_0$ such that
    $$((n+1)^3 - (n+1))((n+1) + 2) = 12j_0.$$

    Set $j_0 = \text{(something)}.$ When we unfactor the equation above, we get

    $$n_0^4 + 6n_0^3 + 11n_0^2 + 6n_0 = 12j_0.$$

    When we substitute in $12k_0$, we get 

    $$12k_0 + 4n_0^3 + 12n_0^2 + 8n_0 = 12j_0.$$

    We can rewrite the statement like this:

    $$12k_0 + 4n_0^3 + 4n_0^2 + 8n_0^2 + 8n_0 = 12j_0.$$

    Utilizing grouping, we can factor out $4n_0(n_0 + 1)$, and are left with

    $$12k_0 + 4n_0(n_0 + 1)(n_0 + 2) = 12j_0.$$

    By lemma 1, $n_0(n_0 + 1)$ is even, and can be rewritten as $2m_0$, with $m_0$ being an integer. When we substitute this in, we get

    $$12k_0 + 8m_0(n_0 + 2) = 12j_0.$$


\end{proof}




% problem #2

\newpage
\begin{problem}{2}
    Let r represent an arbitrary real number other than 0 and 1. Show that for $n \in \mathbb{Z}_0^+$

    $$\sum_{i=0}^{n} r^i = \frac{1 - r^{n+1}}{1 - r}.$$
    This is the formula for what is called the finite geometric series. This formula is quite important in many different fields of mathematics and should be committed to memory.

\end{problem}

\begin{proof} We will prove this using mathematical induction. \\ \\

    Base Case: Set $n_0 = 0.$ Then $$\sum_{i=0}^{n_0} r^i = r^0 = 1 = \frac{1 - r^{n_0+1}}{1 - r}.$$

    This statement is clearly true.

    Inductive Step: Assume there exists a nonnegative integer $n_0$ such that for every real number $r$ aside from 0 and 1,

    $$\sum_{i=0}^{n_0} r^i = \frac{1 - r^{n_0+1}}{1 - r}.$$

    We want to show that this is true for $n+1$, so we need to show that

    $$\sum_{i=0}^{n_0} r^i + r^{n_0+1} = \sum_{i=0}^{n_0 + 1} r^i.$$


\end{proof}


\noindent While working on this proof, I received no external assistance aside from advice from Professor Mehmetaj.

\end{document}
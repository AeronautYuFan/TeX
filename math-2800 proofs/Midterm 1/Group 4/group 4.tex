\documentclass[12pt]{article}
\usepackage[margin=1in]{geometry}
\usepackage{amsmath,amsthm,amssymb}
\newenvironment{theorem}[2][Theorem]{\begin{trivlist}
\item[\hskip \labelsep {\bfseries #1}\hskip \labelsep {\bfseries #2.}]}{\end{trivlist}}
\newenvironment{lemma}[2][Lemma]{\begin{trivlist}
\item[\hskip \labelsep {\bfseries #1}\hskip \labelsep {\bfseries #2.}]}{\end{trivlist}}
\newenvironment{exercise}[2][Exercise]{\begin{trivlist}
\item[\hskip \labelsep {\bfseries #1}\hskip \labelsep {\bfseries #2.}]}{\end{trivlist}}
\newenvironment{problem}[2][Problem]{\begin{trivlist}
\item[\hskip \labelsep {\bfseries #1}\hskip \labelsep {\bfseries #2.}]}{\end{trivlist}}
\newenvironment{question}[2][Question]{\begin{trivlist}
\item[\hskip \labelsep {\bfseries #1}\hskip \labelsep {\bfseries #2.}]}{\end{trivlist}}
\newenvironment{corollary}[2][Corollary]{\begin{trivlist}
\item[\hskip \labelsep {\bfseries #1}\hskip \labelsep {\bfseries #2.}]}{\end{trivlist}}
% Solutions use a modified proof environment
\newenvironment{solution}
{\let\oldqedsymbol=\qedsymbol
\renewcommand{\qedsymbol}{$\blacktriangleleft$}
\begin{proof}[\textit\upshape Solution]}
{\end{proof}
\renewcommand{\qedsymbol}{\oldqedsymbol}}
\begin{document}
% --------------------------------------------------------------
% Start here
% --------------------------------------------------------------
\title{Group 4}%replace X with the appropriate number
\author{Helen Freedman, Pruthvi Jasty, Yu Fan Mei\\ %replace with your name
Introduction to Proof and Problem Solving} %if necessary, replace with your course title
\maketitle
\begin{problem}{4} Consider the following statements

    $$P_1 (f) \equiv \left\{ \forall M \in \mathbb{R}, \exists x \in \mathbb{R} \text{ such that } f(x) > M \right\}$$

    $$P_6 (f) \equiv \left\{ \exists (M, K) \in \mathbb{R}^2 \text{ such that } \forall x > K, f(x) > M \right\}.$$

\noindent    (a) Prove or disprove
\begin{equation} \label{eq}
    \{ \forall f \text{ satisfying } P_1, f \text{ satisfies } P_6 \}.
\end{equation}
     (b) Prove or disprove
\begin{equation} \label{eq}
    \{ \forall f \text{ satisfying } P_6, f \text{ satisfies } P_1 \}.
\end{equation}
\end{problem}

\begin{proof} 

(a) We will disprove the statement that for every function $f$ satisfying $P_1$, $f$ also satisfies $P_6$ by proving that its negation is true. The negation is:

$$\{ \exists f \text{ satisfying } P_1 \text{ such that } f \text{ satisfies } \lnot P_6 \}.$$

Set $f_0 = -x_0$. We will first show that $f_0$ satisfies $P_1$. Let $M_0$ be any real number. Set $x_0 = -M_0 - 1.$ Then,

$$f_0(x_0) = -(-M_0 - 1).$$

Distributing the negative, we get 

$$f_0(x_0) = M_0 + 1.$$

We know that $1 > 0$, so adding $M_0$ to both sides, we get 

$$M_0 + 1 > M_0.$$

Then,

$$f_0(x_0) > M_0.$$

\newpage

Thus, $f_0$ satisfies $P_1$. Now we will prove $f_0$ satisfies $\lnot P_6.$ The negation of $P_6$ is

$$\lnot P_6 \equiv \{ \forall (M, K) \in \mathbb{R}^2, \exists x > K \text{ such that } f(x) \leq M \}. $$

Let $(M_0, K_0)$ be any ordered pair in $\mathbb{R}^2.$ Set $x_0 = | K_0 | + | M_0 | + 1$ Since $|M_0|$ is always nonnegative, we know that

$$| M_0 | + 1 > 0.$$

Then $x_0 > K_0.$ Plugging $x_0$ into $f_0$, we get

$$f_0(x_0) = -(| K_0 | + | M_0 | + 1).$$

Distributing the negative, we get

$$f_0(x_0) = -| K_0 | - | M_0 | - 1.$$

We know $|M_0| \geq M_0$. Multiplying both sides by $-1$, we get

$$-|M_0| \leq M_0.$$

Since we know $1 > 0,$ the inequality still holds true when we subtract $1$ from the left hand side of the inequality. Since $|K_0| \geq 0$, we can apply the same principle here and subtract $|K_0|$ from the left hand side. Then, we get

$$-|K_0| - |M_0| - 1 \leq M_0.$$

Since we know $f_0(x_0) = -| K_0 | - | M_0 | - 1$, we have

$$f_0(x_0) \leq M_0.$$

Thus, we have proven that there exists a $f_0$ that satisfies $\lnot P_6$ and $P_1$, proving the negation of (1) to be true.


\end{proof}

%%%%%%%%%%%%%%%%%%%%%%%%%%%%%%%%%%%%%%%%%%%%%%%%%%%%%%%%%%%%%%%%%%%%%%
\newpage

\begin{proof}

(b) We will disprove the statement that for every function $f$ satisfying $P_6$, $f$ satisfies $P_1$ by proving that the negation is true. The negation of this statement is 

$$\{ \exists f \text{ satisfying } P_6 \text{ such that } f \text{ satisfies } \lnot P_1 \}.$$


Set $f_0(x)=3$. We will first show that $f_0$ satisfies $P_6$. Set $(M_0, K_0)=(2,0)$. Let $x_0$ be any real number greater than $K_0$. We know that $x_0 > 0$. Then,

$$f_0(x_0)=3>2.$$

Next, we will show that $f_0$ satisfies $\lnot P_1$. Set $M_0=4$. Let $x_0$ be any real number. Then,

$$f_0(x_0)=3 \leq 4.$$

Thus, $f_0$ satisfies $\lnot P_1$. Since $f_0$ satisfies both $P_6$ and $\lnot P_1$, we disproved (2) by proving that the negation of (2) is true.

\end{proof}
%Note 2: Inside the align environment, you do not want to use $-signs. The reason for this is that this is
%already a math environment. This is why we have to include \text{} around any text inside the align environment.

\newpage
\begin{problem}{10}
A function $f$ with domain $\mathbb{R}$ mapping into $\mathbb{R}$ is increasing if whenever $a < b,$ then $f(a) < f(b).$ In the following, $f$ and $g$ are assumed
to have domain $D = \mathbb{R}$.

(d) Prove or disprove that if $f$ and $g$ are increasing functions into $\mathbb{R}$, then the composition of $f$ and $g$, $h = f \circ g$ is an increasing function into $\mathbb{R}$.

(e) Prove or disprove that if $f$ and $g$ are increasing functions into $\mathbb{R}$,
then $h = fg$ is an increasing function into $\mathbb{R}$.

\end{problem}

\begin{proof} (d)
We will prove this statement is true. Let $g_{0}$ be any increasing function into $\mathbb{R}.$ By definition, for all $a_0 \in \mathbb{R}$ and $b_0 \in \mathbb{R}$ where $a_{0} < b_{0},$ $g_{0}(a_{0}) < g_{0}(b_{0}).$ 
Let $f_{0}$ be any increasing function into $\mathbb{R}$. By definition, for all $c_{0}$ and $d_{0} \in \mathbb{R}$, where $c_{0} < d_{0},$ $f_{0}(c_{0}) < f_{0}(d_{0}).$ 
Set $m_{0}= g_{0}(a_{0})$ and set $n_{0} = g_{0}(b_{0}).$ By definition, $m_{0} < n_{0}.$ Then, knowing that $f_{0}$ is an increasing function, $$f_{0}(m_{0}) < f_{0}(n_{0}).$$ Plugging $g_{0}(a_{0})$ and $g_{0}(b_{0})$ back into the equation we get that, $$f_0(g_{0}(a_{0})) < f_0(g_{0}(b_{0}))$$ when $g_{0}(a_{0}) < g_{0}(b_{0}).$ As this is the definition of an increasing function into $\mathbb{R}$, we know that $h_0 = f_{0} \circ g_{0}$ must be an increasing function into $\mathbb{R}.$

\end{proof}
\begin{proof}

(e) We will disprove that if $f$ and $g$ are increasing functions into $\mathbb{R}$, then $h = fg$ is an increasing function into $\mathbb{R}$. Set $f_{0}(x_{0}) = x_{0}$ and set $g_{0}(x_{0}) = x_{0}.$ First we will prove that these two functions are increasing functions by showing that for every real number $a$ and $b$ where $b > a$, $f_0(b) > f_0(a)$ and $g_0(b) > g_0(a).$
\\ \\
Let $a_0$ be any real number. Let $b_0$ be any real number greater than $a_0$. Plugging $a_0$ into $f_0$, we get $f_0(a_0) = a_0$. Plugging $b_0$ into $f_0$, we get $f_0(b_0) = b_0$. But we know that

$$b_0 > a_0$$
$$f_0(b_0) > f_0(a_0).$$

\noindent Thus, $f_0$ is an increasing function. Plugging $a_0$ into $g_0$, we get $g_0(a_0) = a_0$. Plugging $b_0$ into $g_0$, we get $g_0(b_0) = b_0$. We know that

$$b_0 > a_0$$
$$g_0(b_0) > g_0(a_0).$$

\noindent Thus, $g_0$ is also an increasing function. Set $h_0(x) = f_0(x) g_0(x)$. Then $h_0(x) = x^2.$ Suppose $h_0(x)$ is an increasing function. That would mean for every real number $a$ and $b$ where $b > a$, $h_0(b) > h_0(a)$. This statement should then hold true for $a_0 = -2$ and $b_0 = -1$, which is also greater than $a_0$. But substitution gives us $h_0(a_0) = 4$ and $h_0(b_0) = 1$. Since 1 is not greater than 4, we have proven that the original statement is false.

\end{proof}

% --------------------------------------------------------------
% You don't have to mess with anything below this line.
% --------------------------------------------------------------
While working on this assignment, we did not receive any outside help except from Professor Mehmetaj.

\end{document}

\documentclass{article}

\usepackage[margin=0.75in]{geometry}
\usepackage{amsmath,amssymb}
\usepackage{graphicx,float}
\usepackage{multirow,setspace}
\usepackage{natbib,enumerate}
\usepackage{caption}
\usepackage{subcaption}
\usepackage{termcal} 
\usepackage{xcolor}
\usepackage{enumitem}
\usepackage{gensymb}
\usepackage{booktabs}
\usepackage{listings, makecell}
\usepackage{hyperref}
\usepackage{multicol}

\setlength{\marginparwidth}{2.5cm}
\newcommand{\HRule}{\rule{\linewidth}{0.5mm}}
\setlength\parindent{0pt}

\newenvironment{problem}[2][Problem]{\begin{trivlist}
\item[\hskip \labelsep {\bfseries #1}\hskip \labelsep {\bfseries #2.}]}{\end{trivlist}}


\begin{document}

    \begin{center}
        \HRule \\[0.4cm]        
        {\Large \textbf{Abstract Algebra HW3}} \\[0.2cm]

        {\large Yu Fan Mei $\cdot$ MATH-3210} \\
        \HRule \\[0.3cm]
    \end{center}

    \textbf{2.} Find the order of each of the following elements.

    \begin{problem}{2.1}
        $5 \in \mathbb{Z}_{12}$.

        \begin{itemize}
            \item The order of $5 \in \mathbb{Z}_{12}$ is $\frac{12}{\texttt{gcd}(5, 12)} = \frac{12}{1} = 12$.
        \end{itemize}
    \end{problem}

    \begin{problem}{2.2}
        $\sqrt{3} \in \mathbb{R}$.

        \begin{itemize}
            \item We will prove that the order is infinite via contradiction. Suppose the order of $\sqrt{3}$ is finite. This would mean there exists a positive integer $k$ such that $k\sqrt{3} = 0$.
            \item Dividing both sides of this equality by $\sqrt{3}$, we get $k = 0$, which contradicts the statement that $k > 0$. This means $\sqrt{3}$ must be infinite.
        \end{itemize}
    \end{problem}

    \begin{problem}{2.3}
        $\sqrt{3} \in \mathbb{R}*$.

        \begin{itemize}
            \item We will prove that the order is infinite, also via contradiction. Suppose the order of $\sqrt{3}$ in the group $\mathbb{R}*$ was finite. Then there exists a positive integer $k$ such that $\sqrt{3}^k = 1$.
            \item Taking the natural logarithm of both sides, we get $k\ln{\sqrt{3}} = 0$. From this, we get $k = 0$, which is a contradiction.
        \end{itemize}
    \end{problem}

    \begin{problem}{2.4} $-i \in \mathbb{C}*$.
        
        \begin{itemize}
            \item The order of $-i$ in the group of complex numbers under multiplication is 4.
            \item $(-i)^2 = -1$, and $(-i)^3 = i$. $(-i)^4 = 1$.
        \end{itemize}
    \end{problem}

    \begin{problem}{2.5} $72 \in \mathbb{Z}_{240}$.
        \begin{itemize}
            \item The order of $72 \in \mathbb{Z}_{240}$ is $\frac{240}{\texttt{gcd}(240, 72)}$.
            \item The gcd of 240 and 72 is 24:
            
            \begin{align*}
                240 & = 72(3) + 24 \\
                72 & = 24(3) + 0.
            \end{align*}

            \item So, the order of $72 \in \mathbb{Z}_{240}$ is $\frac{240}{24} = 10$.
        \end{itemize}
    \end{problem}

 
    \begin{problem}{2.6} $312 \in \mathbb{Z}_{471}$.
        \begin{itemize}
            \item The order of 312 in $\mathbb{Z}_{471}$ is $\frac{471}{\texttt{gcd}(471, 312)}$.
            \item And the greatest common divisor of 471 and 312 is  1:
            
                \begin{align*}
                    471 & = 312(1) + 59 \\
                    312 & = 59(5) + 17 \\
                    59 & = 17(3) + 8 \\
                    17 & = 8(2) + 1 \\
                    8 & = 1(8) + 0.
                \end{align*}

            \item This means the order of $312 \in \mathbb{Z}_{471}$ is 471.
            \end{itemize}
    \end{problem}

    \newpage
    \textbf{3.} List all of the elements in each of the following subgroups.

    \begin{problem}{3.1} The subgroup of $\mathbb{Z}$ generated by 7.
        \begin{itemize}
            \item The elements in this subgroup are $\{..., -14, -7, 0, 7, 14, ...\}.$
            \item This subgroup is infinite.
        \end{itemize}
    \end{problem}

    \begin{problem}{3.2} The subgroup of $\mathbb{Z}_{24}$ generated by 15.
        \begin{itemize}
            \item The subgroup is $\{15, 6, 21, 12, 3, 18, 9, 0\}$.
        \end{itemize}
    \end{problem}

    \begin{problem}{3.3} All subgroups of $\mathbb{Z}_{12}$.
        \begin{itemize}
            \item The subgroups of $\mathbb{Z}_{12}$ are:
            \begin{itemize}
                \item $\langle 0 \rangle = \{0\}$
                \item $\langle 1 \rangle = \{1, 2, 3, ..., 11, 0\} $
                \item $\langle 2 \rangle = \{2, 4, 6, 8, 10, 0 \} $
                \item $\langle 3 \rangle = \{3, 6, 9, 0\} $
                \item $\langle 4 \rangle = \{4, 8, 0\} $
                \item $\langle 6 \rangle = \{6, 0\} $
            \end{itemize}
        \end{itemize}
    \end{problem}


    \begin{problem}{5} Find the order of every element in $\mathbb{Z}_{18}$.
        \begin{itemize}
            \item The order of any element $k \in \mathbb{Z}_{18}$ is given by $\frac{18}{\texttt{gcd}(k, 18)}$.
        \end{itemize}

        \begin{multicols}{3}
            \begin{itemize}
                \item ord(1) = 18
                \item ord(2) = 9
                \item ord(3) = 6
                \item ord(4) = 9
                \item ord(5) = 18
                \item ord(6) = 3
                \item ord(7) = 18
                \item ord(8) = 9
                \item ord(9) = 2
                \item ord(10) = 9
                \item ord(11) = 18
                \item ord(12) = 3
                \item ord(13) = 18
                \item ord(14) = 9
                \item ord(15) = 6
                \item ord(16) = 9
                \item ord(17) = 18
                \item ord(0) = 1
            \end{itemize}
        \end{multicols}
    \end{problem}

    \begin{problem}{26} Prove that $\mathbb{Z}_p$ has no nontrivial subgroups if $p$ is prime.
        \begin{itemize}
            \item Let $\mathbb{Z}_{p_0}$ be the additive group of integers mod $p_0$ such that $p_0$ is a prime integer.
            \item Let $H_0$ be any subgroup of $\mathbb{Z}_{p_0}$. Since every subgroup of a cyclic group is also cyclic, this means there exists an integer $k_0 \in \mathbb{Z}_{p_0}$ such that $k_0$ generates $H_0$, or in other words, $\langle k_0 \rangle = H_0$. 
            \item Because $\mathbb{Z}_{p_0}$ is a modulus group, we know $0 \leq k_0 < p_0$. If $k_0 = 0$, this means $H_0 = \{0\}$, which is a trivial subgroup. Let's suppose $0 < k_0 < p_0$. Then the order of $k_0$ is
            \begin{align*}
                ord(k_0) & = \frac{p_0}{\texttt{gcd}(k_0, p_0)} \\
                    & = \frac{p_0}{1} \\
                    & = p_0.
            \end{align*}

            \item The above is true because $p_0$ is prime, so its greatest common divisor with $k_0$ is 1 because $0 < k_0 < p_0$. Since the order of $k_0$ is $p_0$, this means $|H_0| = |\mathbb{Z}_{p_0}|$.
            \item Following from this and using the fact that $H_0$ is a subgroup, this must mean $H_0$ is the subgroup with all the elements in $\mathbb{Z}_{p_0}$, which is a trivial subgroup.
        \end{itemize}
    \end{problem}






    \newpage
    \begin{problem}{28} Let $a$ be a generator in group $G$. What is a generator for $\langle a^m \rangle \cap \langle a^n \rangle$?

        \begin{itemize}
            \item A generator for $\langle a^m \rangle \cap \langle a^n \rangle$ would be $\langle a^{\texttt{lcm}(m, n)} \rangle$, and we'll show this by proving equality of subgroups.
            \item Set $x_0 = \texttt{lcm}(m, n)$, and set $H = \langle a^{x_0} \rangle$. We will prove $H \subseteq \langle a^m \rangle \cap \langle a^n \rangle$ and $H \supseteq  \langle a^m \rangle \cap \langle a^n \rangle$. Because of how we defined $x_0$, there exist integers $k_1, k_2$ such that $x_0 = k_1m = k_2n$. 
            \item Let $h_0 \in H$. Then there exists $l \in \mathbb{Z}$ such that $h_0 = a^{x_0l}$. Since $x_0 = k_1m = k_2n$, we can see that $h_0 = a^{k_1lm} \in \langle a^m \rangle$ and $h_0 = a^{k_2ln} \in \langle a^n \rangle$. Thus, $H \subseteq \langle a^m \rangle \cap \langle a^n \rangle$.
            \item Let $j_0 \in \langle a^m \rangle \cap \langle a^n \rangle$. Then there exist integers $s_1, s_2$ such that $j_0 = a^{ms_1} = a^{ns_2}$. This means $ms_1 = ns_2$. But this also means $j_0 = a^{mx_0t}$, where $t \in \mathbb{Z}$.
            \item Rewriting the last equality, we can see that $j_0 = (a^{x_0})^{mt}$. This means $j_0 \in H$, and thus $a^{\texttt{lcm}(m, n)}$ is a generator.
        \end{itemize}
        
    \end{problem}

    \begin{problem}{30} Suppose $G$ is a group and let $a, b \in G$. Prove that if $|a| = m$ and $|b| = n$ with $\texttt{gcd}(m, n) = 1$, then $\langle a \rangle \cap \langle b \rangle = \{ e \}$.

    \end{problem}




































\end{document}
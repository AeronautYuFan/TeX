
% --------------------------------------------------------------
% This is all preamble stuff that you don't have to worry about.
% Head down to where it says "Start here"
% --------------------------------------------------------------

\documentclass[12pt]{article}
 
\usepackage[margin=1in]{geometry} 
\usepackage{amsmath,amsthm,amssymb}
 

 
\newenvironment{theorem}[2][Theorem]{\begin{trivlist}
\item[\hskip \labelsep {\bfseries #1}\hskip \labelsep {\bfseries #2.}]}{\end{trivlist}}
\newenvironment{lemma}[2][Lemma]{\begin{trivlist}
\item[\hskip \labelsep {\bfseries #1}\hskip \labelsep {\bfseries #2.}]}{\end{trivlist}}
\newenvironment{exercise}[2][Exercise]{\begin{trivlist}
\item[\hskip \labelsep {\bfseries #1}\hskip \labelsep {\bfseries #2.}]}{\end{trivlist}}
\newenvironment{problem}[2][Problem]{\begin{trivlist}
\item[\hskip \labelsep {\bfseries #1}\hskip \labelsep {\bfseries #2.}]}{\end{trivlist}}
\newenvironment{question}[2][Question]{\begin{trivlist}
\item[\hskip \labelsep {\bfseries #1}\hskip \labelsep {\bfseries #2.}]}{\end{trivlist}}
\newenvironment{corollary}[2][Corollary]{\begin{trivlist}
\item[\hskip \labelsep {\bfseries #1}\hskip \labelsep {\bfseries #2.}]}{\end{trivlist}}

% Solutions use a modified proof environment
\newenvironment{solution}
               {\let\oldqedsymbol=\qedsymbol
                \renewcommand{\qedsymbol}{$\blacktriangleleft$}
                \begin{proof}[\textit\upshape Solution]}
               {\end{proof}
                \renewcommand{\qedsymbol}{\oldqedsymbol}}
 
\begin{document}
 
% --------------------------------------------------------------
%                         Start here
% --------------------------------------------------------------
 
\title{Individual 1}%replace X with the appropriate number
\author{Yu Fan Mei\\ %replace with your name
Introduction to Proof and Problem Solving} %if necessary, replace with your course title
 
\maketitle
 
\begin{theorem}{x.yz} %You can use theorem, exercise, problem, or question here.  Modify x.yz to be whatever number you are proving
Delete this text and write theorem statement here.
\end{theorem}
 
\begin{proof}
Blah, blah, blah.  Here is an example of the \texttt{align} environment:
%Note 1: The * tells LaTeX not to number the lines.  If you remove the *, be sure to remove it below, too.
%Note 2: Inside the align environment, you do not want to use $-signs.  The reason for this is that this is already a math environment. This is why we have to include \text{} around any text inside the align environment.
\begin{align*}
\sum_{i=1}^{k+1}i & = \left(\sum_{i=1}^{k}i\right) +(k+1)\\ 
& = \frac{k(k+1)}{2}+k+1 & (\text{by inductive hypothesis})\\
& = \frac{k(k+1)+2(k+1)}{2}\\
& = \frac{(k+1)(k+2)}{2}\\
& = \frac{(k+1)((k+1)+1)}{2}.
\end{align*}
\end{proof}
 
\begin{theorem}{x.yz}
Let $n\in \mathbb{Z}$.  Then yada yada. 
\end{theorem}
 
\begin{proof}
Blah, blah, blah.  I'm so smart. 
\end{proof}

\begin{problem}{1}
Create and fill in a truth table for the logical formula
\begin{equation} \label{eq}
((P \vee Q) \Longrightarrow R) \Longrightarrow ((P \Longrightarrow R) \wedge (Q \Longrightarrow R)).
\end{equation}
You probably will want to use the \texttt{tabular} or \texttt{array} environment, as well as the \texttt{table} environment (to add a caption, for example).  Your truth table should have at least four columns (for the propositions $P$, $Q$, $R$ and the large formula (\ref{eq})) and nine rows (one for each possible combination of truth values of $P$, $Q$ and $R$ plus a top row labeling the columns).  \emph{Warning}:  Formula (\ref{eq}) may take up too much space if you include it as a column label in your table.  Think carefully about how to resolve this problem.
\end{problem}


\begin{solution} Blah blah blah
\newpage 

\begin{table}[t] %Instead of t, which refers to top, you may also use h for here or b for bottom.
	\centering
	\begin{tabular}{|c|r|c|l|} % c centers the column contents; l makes the column contents left-justified, and r makes the column contents right-justified.
		\hline
		$P$ & $Q$ & $R$ &  $((P \Longrightarrow R) \wedge (Q \Longrightarrow R))$ \\
		\hline
		T & T & T & T\\
		\hline
		& & & \\
		\hline
		& & & \\
		\hline
		& & & \\
		\hline
		& & & \\
		\hline
		& & & \\
		\hline
		& & & \\
		\hline
		& & & \\
		\hline
	\end{tabular}
	\caption{Table with 4 columns and nine rows}
	\label{tab:truth1}
\end{table}

\end{solution}


 
% --------------------------------------------------------------
%     You don't have to mess with anything below this line.
% --------------------------------------------------------------
 
\end{document}

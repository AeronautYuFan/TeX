\documentclass[12pt]{article}
 
\usepackage[margin=1in]{geometry} 
\usepackage{amsmath,amsthm,amssymb}
\usepackage{comment}
 
\newenvironment{theorem}[2][Theorem]{\begin{trivlist}
\item[\hskip \labelsep {\bfseries #1}\hskip \labelsep {\bfseries #2.}]}{\end{trivlist}}
\newenvironment{lemma}[2][Lemma]{\begin{trivlist}
\item[\hskip \labelsep {\bfseries #1}\hskip \labelsep {\bfseries #2.}]}{\end{trivlist}}
\newenvironment{exercise}[2][Exercise]{\begin{trivlist}
\item[\hskip \labelsep {\bfseries #1}\hskip \labelsep {\bfseries #2.}]}{\end{trivlist}}
\newenvironment{problem}[2][Problem]{\begin{trivlist}
\item[\hskip \labelsep {\bfseries #1}\hskip \labelsep {\bfseries #2.}]}{\end{trivlist}}
\newenvironment{question}[2][Question]{\begin{trivlist}
\item[\hskip \labelsep {\bfseries #1}\hskip \labelsep {\bfseries #2.}]}{\end{trivlist}}
\newenvironment{corollary}[2][Corollary]{\begin{trivlist}
\item[\hskip \labelsep {\bfseries #1}\hskip \labelsep {\bfseries #2.}]}{\end{trivlist}}

% Solutions use a modified proof environment
\newenvironment{solution}
               {\let\oldqedsymbol=\qedsymbol
                \renewcommand{\qedsymbol}{$\blacktriangleleft$}
                \begin{proof}[\textit\upshape Solution]}
               {\end{proof}
                \renewcommand{\qedsymbol}{\oldqedsymbol}}
 
\begin{document}

% --------------------------------------------------------------
%                         Start here
% --------------------------------------------------------------

\title{Individual 8}%replace X with the appropriate number
\author{Yu Fan Mei\\
	Introduction to Proof and Problem Solving} %if necessary, replace with your course title

\maketitle

\begin{problem}{1}
    Consider the function
    \begin{equation*}
    f(x) =
    \begin{cases}
    x - 2 & x \leq 4\\
    \frac{3}{2}x - 2 & x > 4
    \end{cases}.
    \end{equation*}
    Show that $\lim_{x\to 4} f(x)$ does not exist.
\end{problem}

\begin{proof}
    We will prove $\lim_{x\to 4} f(x)\neq $ using cases. Set $\epsilon_0 = 2$. Let $\delta_0$ be an arbitrary real number greater than 0.

    We will consider cases:

    \noindent Case 1: $L_0 > 3$. Set $x = x_0 = 4 + 2\delta/3$. Then, $x$ clearly is greater than 0 and less than $\delta_0$:

    $$0 < |x_0 - 4| = |4 + \frac{2\delta_0}{3} - 4| = |\frac{2\delta_0}{3}| = \frac{2\delta_0}{3}.$$.

    Since $x_0 > 4$, 

    \begin{align*}
        |f(x_0) - L_0| & = |\frac{3}{2}x_0 - 2 - L_0| \\
        & = |\frac{3}{2}(4 + \frac{2\delta_0}{3}) - 2 - L_0| \\
        & = |4 + \delta_0 - L_0|.
    \end{align*}

    Since $L_0 > 3$, we know that $-L_0 < 3$. Adding 4 to both sides, we can see that $4-L_0 < 7$. So,

    \[4 + \delta_0 - L_0 >  \]



\end{proof}

\noindent While working on this proof, I received no external assistance aside from advice from Professor Mehmetaj.

\end{document}

% --------------------------------------------------------------
% This is all preamble stuff that you don't have to worry about.
% Head down to where it says "Start here"
% --------------------------------------------------------------

\documentclass[12pt]{article}
 
\usepackage[margin=1in]{geometry} 
\usepackage{amsmath,amsthm,amssymb}
\usepackage{comment}
 

 
\newenvironment{theorem}[2][Theorem]{\begin{trivlist}
\item[\hskip \labelsep {\bfseries #1}\hskip \labelsep {\bfseries #2.}]}{\end{trivlist}}
\newenvironment{lemma}[2][Lemma]{\begin{trivlist}
\item[\hskip \labelsep {\bfseries #1}\hskip \labelsep {\bfseries #2.}]}{\end{trivlist}}
\newenvironment{exercise}[2][Exercise]{\begin{trivlist}
\item[\hskip \labelsep {\bfseries #1}\hskip \labelsep {\bfseries #2.}]}{\end{trivlist}}
\newenvironment{problem}[2][Problem]{\begin{trivlist}
\item[\hskip \labelsep {\bfseries #1}\hskip \labelsep {\bfseries #2.}]}{\end{trivlist}}
\newenvironment{question}[2][Question]{\begin{trivlist}
\item[\hskip \labelsep {\bfseries #1}\hskip \labelsep {\bfseries #2.}]}{\end{trivlist}}
\newenvironment{corollary}[2][Corollary]{\begin{trivlist}
\item[\hskip \labelsep {\bfseries #1}\hskip \labelsep {\bfseries #2.}]}{\end{trivlist}}

% Solutions use a modified proof environment
\newenvironment{solution}
               {\let\oldqedsymbol=\qedsymbol
                \renewcommand{\qedsymbol}{$\blacktriangleleft$}
                \begin{proof}[\textit\upshape Solution]}
               {\end{proof}
                \renewcommand{\qedsymbol}{\oldqedsymbol}}
 
\begin{document}

% --------------------------------------------------------------
%                         Start here
% --------------------------------------------------------------

\title{Individual 6}%replace X with the appropriate number
\author{Yu Fan Mei\\ %replace with your name
	Introduction to Proof and Problem Solving} %if necessary, replace with your course title

\maketitle

\begin{problem}{12} %You can use theorem, exercise, problem, or question here.  Modify x.yz to be whatever number you are proving
    Show the function $f$ mapping $\mathbb{Z}$ into $S = \mathbb{Z} $ is one-to-one or find two integers $n_1$ and $n_2$ such that $n_1 \neq n_2$ but $f(n_1) = f(n_2)$, where

    $$f(n) = 
        \begin{cases} 
        0.5n + 3 & \text{if } n \in E \\ 
        3n - 1 & \text{if } n \in O 
        \end{cases}$$


    
    

\end{problem}

\begin{proof} Suppose this is true. Set $n_1 = -2$ and $n_2 = 1$. Since $n_1$ is even, 
    \begin{align*} 
        f(n_1) & = 0.5n_1 + 3 \\
        f(n_1) & = 2.
    \end{align*}
    
    Since $n_2$ is odd,

    \begin{align*} 
        f(n_2) & = 3n_2 - 1 \\
        f(n_2) & = 2.
    \end{align*}

    We can observe that $f(n_1) = f(n_2)$. Additionally, since $n_1 \neq n_2$, this function is not one-to-one.


\end{proof}


\noindent While working on this proof, I received no external assistance aside from advice from Professor Mehmetaj.

\end{document}
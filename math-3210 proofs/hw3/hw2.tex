\documentclass{article}

\usepackage[margin=0.75in]{geometry}
\usepackage{amsmath,amssymb}
\usepackage{graphicx,float}
\usepackage{multirow,setspace}
\usepackage{natbib,enumerate}
\usepackage{caption}
\usepackage{subcaption}
\usepackage{termcal} 
\usepackage{xcolor}
\usepackage{enumitem}
\usepackage{gensymb}
\usepackage{booktabs}
\usepackage{listings, makecell}
\usepackage{hyperref}

\setlength{\marginparwidth}{2.5cm}
\newcommand{\HRule}{\rule{\linewidth}{0.5mm}}
\setlength\parindent{0pt}

\newenvironment{problem}[2][Problem]{\begin{trivlist}
\item[\hskip \labelsep {\bfseries #1}\hskip \labelsep {\bfseries #2.}]}{\end{trivlist}}


\begin{document}

    \begin{center}
        \HRule \\[0.4cm]        
        {\Large \textbf{Abstract Algebra HW3}} \\[0.2cm]

        {\large Yu Fan Mei $\cdot$ MATH-3210} \\
        \HRule \\[0.3cm]
    \end{center}

    \textbf{2.} Find the order of each of the following elements.

    \begin{problem}{2.1}
        $5 \in \mathbb{Z}_{12}$.

        \begin{itemize}
            \item The order of $5 \in \mathbb{Z}_{12}$ is $\frac{12}{\texttt{gcd}(5, 12)} = \frac{12}{1} = 12$.
        \end{itemize}
    \end{problem}

    \begin{problem}{2.2}
        $\sqrt{3} \in \mathbb{R}$.

        \begin{itemize}
            \item We will prove that the order is infinite via contradiction. Suppose the order of $\sqrt{3}$ is finite. This would mean there exists a positive integer $k$ such that $k\sqrt{3} = 0$.
            \item Dividing both sides of this equality by $\sqrt{3}$, we get $k = 0$, which contradicts the statement that $k > 0$. This means $\sqrt{3}$ must be infinite.
        \end{itemize}
    \end{problem}

    \begin{problem}{2.3}
        $\sqrt{3} \in \mathbb{R}*$.

        \begin{itemize}
            \item We will prove that the order is infinite, also via contradiction. Suppose the order of $\sqrt{3}$ in the group $\mathbb{R}*$ was finite. Then there exists a positive integer $k$ such that $\sqrt{3}^k = 1$.
            \item Taking the natural logarithm of both sides, we get $k\ln{\sqrt{3}} = 0$. From this, we get $k = 0$, which is a contradiction.
        \end{itemize}
    \end{problem}

    \begin{problem}{2.4} $-i \in \mathbb{C}*$.
        
        \begin{itemize}
            \item The order of $-i$ in the group of complex numbers under multiplication is 4.
            \item $(-i)^2 = -1$, and $(-i)^3 = i$. $(-i)^4 = 1$.
        \end{itemize}
    \end{problem}

    \begin{problem}{2.5} $72 \in \mathbb{Z}_{240}$.
        \begin{itemize}
            \item The order of $72 \in \mathbb{Z}_{240}$ is $\frac{240}{\texttt{gcd}(240, 72)}$.
            \item The gcd of 240 and 72 is 24:
            
            \begin{align*}
                240 & = 72(3) + 24 \\
                72 & = 24(3) + 0.
            \end{align*}

            \item So, the order of $72 \in \mathbb{Z}_{240}$ is $\frac{240}{24} = 10$.
        \end{itemize}
    \end{problem}

 
    \begin{problem}{2.6} $312 \in \mathbb{Z}_{471}$.
        \begin{itemize}
            \item The order of 312 in $\mathbb{Z}_{471}$ is $\frac{471}{\texttt{gcd}(471, 312)}$.
            \item And the greatest common divisor of 471 and 312 is  1:
            
                \begin{align*}
                    471 & = 312(1) + 59 \\
                    312 & = 59(5) + 17 \\
                    59 & = 17(3) + 8 \\
                    17 & = 8(2) + 1 \\
                    8 & = 1(8) + 0.
                \end{align*}

            \item This means the order of $312 \in \mathbb{Z}_{471}$ is 471.
            \end{itemize}
    \end{problem}

    \newpage
    \textbf{3.} List all of the elements in each of the following subgroups.

    \begin{problem}{3.1} The subgroup of $\mathbb{Z}$ generated by 7.
        \begin{itemize}
            \item The elements in this subgroup are $\{..., -14, -7, 0, 7, 14, ...\}.$
            \item This subgroup is infinite.
        \end{itemize}
    \end{problem}

    \begin{problem}{3.2} The subgroup of $\mathbb{Z}_{24}$ generated by 15.
        \begin{itemize}
            \item The subgroup is $\{15, 6, 21, 12, 3, 18, 9, 0\}$.
        \end{itemize}
    \end{problem}

    \begin{problem}{3.3} All subgroups of $\mathbb{Z}_{12}$.
        \begin{itemize}
            \item The subgroups of $\mathbb{Z}_{12}$ are:
            \begin{itemize}
                \item $\langle 0 \rangle = \{0\}$
                \item $\langle 1 \rangle = \{1, 2, 3, ..., 11, 0\} $
                \item $\langle 2 \rangle = \{2, 4, 6, 8, 10, 0 \} $
                \item $\langle 3 \rangle = \{3, 6, 9, 0\} $
                \item $\langle 4 \rangle = \{4, 8, 0\} $
                \item $\langle 6 \rangle = \{6, 0\} $
            \end{itemize}
        \end{itemize}
    \end{problem}


    \begin{problem}{5} Find the order of every element in $\mathbb{Z}_{18}$.
        
    \end{problem}




































\end{document}
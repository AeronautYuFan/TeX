\documentclass[12pt]{article}
 
\usepackage[margin=1in]{geometry} 
\usepackage{amsmath,amsthm,amssymb}
\usepackage{comment}
 
\newenvironment{theorem}[2][Theorem]{\begin{trivlist}
\item[\hskip \labelsep {\bfseries #1}\hskip \labelsep {\bfseries #2.}]}{\end{trivlist}}
\newenvironment{lemma}[2][Lemma]{\begin{trivlist}
\item[\hskip \labelsep {\bfseries #1}\hskip \labelsep {\bfseries #2.}]}{\end{trivlist}}
\newenvironment{exercise}[2][Exercise]{\begin{trivlist}
\item[\hskip \labelsep {\bfseries #1}\hskip \labelsep {\bfseries #2.}]}{\end{trivlist}}
\newenvironment{problem}[2][Problem]{\begin{trivlist}
\item[\hskip \labelsep {\bfseries #1}\hskip \labelsep {\bfseries #2.}]}{\end{trivlist}}
\newenvironment{question}[2][Question]{\begin{trivlist}
\item[\hskip \labelsep {\bfseries #1}\hskip \labelsep {\bfseries #2.}]}{\end{trivlist}}
\newenvironment{corollary}[2][Corollary]{\begin{trivlist}
\item[\hskip \labelsep {\bfseries #1}\hskip \labelsep {\bfseries #2.}]}{\end{trivlist}}

% Solutions use a modified proof environment
\newenvironment{solution}
               {\let\oldqedsymbol=\qedsymbol
                \renewcommand{\qedsymbol}{$\blacktriangleleft$}
                \begin{proof}[\textit\upshape Solution]}
               {\end{proof}
                \renewcommand{\qedsymbol}{\oldqedsymbol}}
 
\begin{document}

% --------------------------------------------------------------
%                         Start here
% --------------------------------------------------------------

\title{Group 9}%replace X with the appropriate number
\author{Sean Clavadetscher and Yu Fan Mei\\
	Introduction to Proof and Problem Solving} %if necessary, replace with your course title

\maketitle

\begin{problem}{6}
    Define the relation \(F\) on \(\mathbb{R}^2\) by  
\[
((x, y), (s, t)) \in F \iff x^2 - y = s^2 - t.
\]

(Hint: Make sure you work enough examples to understand what pairs of points are related and why.) 

Show \(F\) is an equivalence relation or give a counterexample to one of the properties of an equivalence relation. If \(F\) is an equivalence relation, describe the equivalence classes for \(F\).
\end{problem}

\begin{proof}
    We will prove that $F$ is an equivalence relation by proving that it is reflexive, symmetric, and transitive. We will first show that $F$ is reflexive:

    Let $(x_0, y_0)$ be any point in $\mathbb{R}^2$. Then it follows that
    $$x_0^2 - y_0^2 = x_0^2 - y_0^2.$$
    
    \noindent
    Then $(x_0, y_0)F(x_0, y_0)$ holds true, and $F$ is reflexive. Next, we must prove that $F$ is symmetric:

    Let $(x_0, y_0)$ and $(s_0, t_0)$ be any points in $\mathbb{R}^2$ such that $(x_0, y_0)F(s_0, t_0)$. By definition, $x_0^2 - y_0 = s_0^2 - t_0$. Then we know that 
    
    $$s_0^2 - t_0 = x_0^2 - y_0.$$

    \noindent
    This shows that $(s_0, t_0)F(x_0, y_0)$, and $F$ is symmetric. Finally, we must prove that $F$ is transitive:

    Let $(x_0, y_0)$, $(s_0, t_0)$, and $(a_0, b_0)$ be any points in $\mathbb{R}^2$ such that $(x_0, y_0)F(s_0, t_0)$ and $(s_0, t_0)F(a_0, b_0)$. By definition, $x_0^2 - y_0 = s_0^2 - t_0$ and $s_0^2 - t_0 = a_0^2 - b_0$. Then it follows that
    
    $$x_0^2 - y_0 = a_0^2 - b_0.$$

    By the definition of the relation, this means that $(x_0, y_0)F(a_0, b_0)$, and $F$ is transitive. Since $F$ is reflexive, symmetric, and transitive, $F$ is an equivalence relation.

\end{proof}

\newpage
\begin{problem}{8}
    Let \( x, y \in \mathbb{R}^+ \).

\noindent Define the relation \( A \) such that \( (x, y) \in A \) if and only if there exists \( n \in \mathbb{Z}^+_0 \) (the set of non-negative integers) such that \( x = 2^n y \). Is \( A \) an equivalence relation? Explain.
\end{problem}

\begin{proof}
    We will first prove that
    
    \end{proof}

\noindent While working on this proof, we received no external assistance aside from advice from Professor Mehmetaj.

\end{document}
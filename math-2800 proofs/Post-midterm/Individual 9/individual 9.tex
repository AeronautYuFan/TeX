\documentclass[12pt]{article}
 
\usepackage[margin=1in]{geometry} 
\usepackage{amsmath,amsthm,amssymb}
\usepackage{comment}
 
\newenvironment{theorem}[2][Theorem]{\begin{trivlist}
\item[\hskip \labelsep {\bfseries #1}\hskip \labelsep {\bfseries #2.}]}{\end{trivlist}}
\newenvironment{lemma}[2][Lemma]{\begin{trivlist}
\item[\hskip \labelsep {\bfseries #1}\hskip \labelsep {\bfseries #2.}]}{\end{trivlist}}
\newenvironment{exercise}[2][Exercise]{\begin{trivlist}
\item[\hskip \labelsep {\bfseries #1}\hskip \labelsep {\bfseries #2.}]}{\end{trivlist}}
\newenvironment{problem}[2][Problem]{\begin{trivlist}
\item[\hskip \labelsep {\bfseries #1}\hskip \labelsep {\bfseries #2.}]}{\end{trivlist}}
\newenvironment{question}[2][Question]{\begin{trivlist}
\item[\hskip \labelsep {\bfseries #1}\hskip \labelsep {\bfseries #2.}]}{\end{trivlist}}
\newenvironment{corollary}[2][Corollary]{\begin{trivlist}
\item[\hskip \labelsep {\bfseries #1}\hskip \labelsep {\bfseries #2.}]}{\end{trivlist}}

% Solutions use a modified proof environment
\newenvironment{solution}
               {\let\oldqedsymbol=\qedsymbol
                \renewcommand{\qedsymbol}{$\blacktriangleleft$}
                \begin{proof}[\textit\upshape Solution]}
               {\end{proof}
                \renewcommand{\qedsymbol}{\oldqedsymbol}}
 
\begin{document}

% --------------------------------------------------------------
%                         Start here
% --------------------------------------------------------------

\title{Individual 9}%replace X with the appropriate number
\author{Yu Fan Mei\\
	Introduction to Proof and Problem Solving} %if necessary, replace with your course title

\maketitle

\begin{problem}{1}
    Show that  
\[
C((0, 1]) = C((0, 1))
\]  
by showing that the function  
\[
f(x) = 
\begin{cases} 
x & \text{if } x \neq \frac{1}{n} \text{ for any } n \in \mathbb{Z}^+ \\
\frac{1}{n+1} & \text{if } x = \frac{1}{n} \text{ for some } n \in \mathbb{Z}^+
\end{cases}
\]
is one-to-one from \((0, 1]\) onto \((0, 1)\). It might help to graph the function.
\end{problem}

\begin{proof} In order to prove these two sets have the same cardinality, we will prove that the function $f : (0, 1] \to (0, 1) $ is one-to-one and onto. We will first use a proof by contraposition to prove that $f$ is one-to-one. Let $x_1, x_2$ be any two real numbers within $(0, 1]$ such that $f(x_1) = f(x_2)$. Let's consider cases:
    
    Case 1: Suppose $x_1, x_2 \neq 1/n$ for any $n \in \mathbb{Z}^+$. By definition, $f(x_1) = f(x_2),$ and it then follows that $x_1 = x_2$. 
    
    Case 2: Suppose there exists positive integers $n_1, n_2$ such that $x_1 = 1/n_1$ and $x_2 = 1/n_2$. Since $f(x_1) = f(x_2)$, it follows that

    $$\frac{1}{n_1 + 1} = \frac{1}{n_2 + 1}.$$

    This means that their reciprocals are also equivalent:

    $$n_1 + 1 = n_2 + 1.$$

    Subtracting 1 from both sides, we get $n_1 = n_2$. The reciprocal of both sides of this equation is $1/n_1 = 1/n_2$. From this it follows that $x_1 = x_2$.

    Case 3: Without loss of generality, suppose $x_1 \neq 1/j$ for any $j \in \mathbb{Z}^+$ and there exists a positive integer $n_2$ such that $x_2 = 1/n_2$. Then $f(x_1) = x_1$ and $f(x_2) = 1/(n_2 + 1)$. Set $j_0 = n_2 + 1$. Since integers are closed under addition, $j_0$ is an integer. Then we get

    $$x_1 = \frac{1}{j_0},$$

    which is a contradiction. Since case 3 cannot occur and cases 1 and 2 hold true, we have proven that $f: (0, 1] \to (0, 1)$ is one-to-one.

    \noindent
    We will now show that $f: (0, 1] \to (0, 1) $ is onto. Let $y_0$ be any real number such that $y_0 \in (0, 1)$. We will consider cases again:

    Case 1: Suppose there exists an $n_0 \in \mathbb{Z}^+$ such that $y_0 = 1/(n_0 + 1)$. Set $x_0 = 1/(n_0)$. Then 
    
    $$f(x_0) = \frac{1}{n+1}.$$

    Case 2: Suppose $y_0 \neq 1/n$ for all $n \in \mathbb{Z}^+$. Set $x_0 = y_0$. Then

    $$f(x_0) = x_0 = y_0.$$

    Thus, we have proven that $f: (0, 1] \to (0, 1) $ maps $(0, 1]$ onto $(0, 1)$. Since we've proven that $f: (0, 1] \to (0, 1) $ is both one-to-one and onto, $C((0, 1]) = C((0, 1))$.

\end{proof}


\noindent While working on this proof, I received no external assistance aside from advice from Professor Mehmetaj.

\end{document}

% --------------------------------------------------------------
% This is all preamble stuff that you don't have to worry about.
% Head down to where it says "Start here"
% --------------------------------------------------------------

\documentclass[12pt]{article}
 
\usepackage[margin=1in]{geometry} 
\usepackage{amsmath,amsthm,amssymb}
\usepackage{comment}
 

 
\newenvironment{theorem}[2][Theorem]{\begin{trivlist}
\item[\hskip \labelsep {\bfseries #1}\hskip \labelsep {\bfseries #2.}]}{\end{trivlist}}
\newenvironment{lemma}[2][Lemma]{\begin{trivlist}
\item[\hskip \labelsep {\bfseries #1}\hskip \labelsep {\bfseries #2.}]}{\end{trivlist}}
\newenvironment{exercise}[2][Exercise]{\begin{trivlist}
\item[\hskip \labelsep {\bfseries #1}\hskip \labelsep {\bfseries #2.}]}{\end{trivlist}}
\newenvironment{problem}[2][Problem]{\begin{trivlist}
\item[\hskip \labelsep {\bfseries #1}\hskip \labelsep {\bfseries #2.}]}{\end{trivlist}}
\newenvironment{question}[2][Question]{\begin{trivlist}
\item[\hskip \labelsep {\bfseries #1}\hskip \labelsep {\bfseries #2.}]}{\end{trivlist}}
\newenvironment{corollary}[2][Corollary]{\begin{trivlist}
\item[\hskip \labelsep {\bfseries #1}\hskip \labelsep {\bfseries #2.}]}{\end{trivlist}}

% Solutions use a modified proof environment
\newenvironment{solution}
               {\let\oldqedsymbol=\qedsymbol
                \renewcommand{\qedsymbol}{$\blacktriangleleft$}
                \begin{proof}[\textit\upshape Solution]}
               {\end{proof}
                \renewcommand{\qedsymbol}{\oldqedsymbol}}
 
\begin{document}

% --------------------------------------------------------------
%                         Start here
% --------------------------------------------------------------

\title{Individual 4}%replace X with the appropriate number
\author{Yu Fan Mei\\ %replace with your name
	Introduction to Proof and Problem Solving} %if necessary, replace with your course title

\maketitle

\begin{problem}{12} %You can use theorem, exercise, problem, or question here.  Modify x.yz to be whatever number you are proving
    (1) Consider the statement
    $$p \equiv \{\forall M \in \mathbb{R}, \exists K \in \mathbb{R}\,\, \text{s.t.} \,\,\forall x > K,
    f(x) > M\}.$$
    \begin{enumerate}
    \item[(a)] Write $\neg p$.
    \item[(b)] Consider the function $f$ from $\mathbb{R}$ into $\mathbb{R}$ defined by
    $$f(x) = \begin{cases}
    \frac{1}{x} & x \neq 0\\
    0 & x = 0
    \end{cases}.$$
    \end{enumerate}
    Does $f$ satisfy $p$ or $\neg p$? Prove your answer.
    

\end{problem}

\begin{solution}

    (a) The negation of $p$ is

    $$\lnot p \equiv \{ \exists M \in \mathbb{R} \text{ such that } \forall K \in \mathbb{R}, \exists x > K \text{ such that } f(x) \leq M\}$$
\end{solution}

\begin{proof}
    
    We will prove that $f$ satisfies $\lnot p.$ Set $M_0 = 2$. Let $K_0$ be any real number. Set $x_0 = |K_0| + 3$. Since $|K_0| \geq 0$, we know that $|K_0| + 2 > 0$. Adding 1 to both sides, we get $|K_0| + 3 > 1$, or $x_0 > 1$. \\
    
    \noindent We know that $2 > 1$. Since $x_0 > 1$, multiplying the left hand side of the inequality by $x_0$ gives us $2x_0 > 1$. Diving both sides by $x_0$, we get

    \begin{align*}
        2 & > \frac{1}{x_0} \\
        M_0 & > \frac{1}{x_0}. \\
        M_0 & \geq \frac{1}{x_0}. \\
    \end{align*}
    
    Thus, the negation of $p$ is true.
\end{proof}


\noindent While working on this proof, I received no external assistance aside from advice from Professor Mehmetaj.

\end{document}
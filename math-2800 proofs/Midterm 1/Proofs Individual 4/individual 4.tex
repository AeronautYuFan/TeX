
% --------------------------------------------------------------
% This is all preamble stuff that you don't have to worry about.
% Head down to where it says "Start here"
% --------------------------------------------------------------

\documentclass[12pt]{article}
 
\usepackage[margin=1in]{geometry} 
\usepackage{amsmath,amsthm,amssymb}
\usepackage{comment}
 

 
\newenvironment{theorem}[2][Theorem]{\begin{trivlist}
\item[\hskip \labelsep {\bfseries #1}\hskip \labelsep {\bfseries #2.}]}{\end{trivlist}}
\newenvironment{lemma}[2][Lemma]{\begin{trivlist}
\item[\hskip \labelsep {\bfseries #1}\hskip \labelsep {\bfseries #2.}]}{\end{trivlist}}
\newenvironment{exercise}[2][Exercise]{\begin{trivlist}
\item[\hskip \labelsep {\bfseries #1}\hskip \labelsep {\bfseries #2.}]}{\end{trivlist}}
\newenvironment{problem}[2][Problem]{\begin{trivlist}
\item[\hskip \labelsep {\bfseries #1}\hskip \labelsep {\bfseries #2.}]}{\end{trivlist}}
\newenvironment{question}[2][Question]{\begin{trivlist}
\item[\hskip \labelsep {\bfseries #1}\hskip \labelsep {\bfseries #2.}]}{\end{trivlist}}
\newenvironment{corollary}[2][Corollary]{\begin{trivlist}
\item[\hskip \labelsep {\bfseries #1}\hskip \labelsep {\bfseries #2.}]}{\end{trivlist}}

% Solutions use a modified proof environment
\newenvironment{solution}
               {\let\oldqedsymbol=\qedsymbol
                \renewcommand{\qedsymbol}{$\blacktriangleleft$}
                \begin{proof}[\textit\upshape Solution]}
               {\end{proof}
                \renewcommand{\qedsymbol}{\oldqedsymbol}}
 
\begin{document}

% --------------------------------------------------------------
%                         Start here
% --------------------------------------------------------------

\title{Individual 4}%replace X with the appropriate number
\author{Yu Fan Mei\\ %replace with your name
	Introduction to Proof and Problem Solving} %if necessary, replace with your course title

\maketitle

\begin{problem}{12} %You can use theorem, exercise, problem, or question here.  Modify x.yz to be whatever number you are proving
Write the negation of the statement

$$p \equiv \{ \forall n \in \mathbb{Z}, \exists m \in \mathbb{Z} \text{ such that } 4n + 3m = 0 \text{ or } 4n + 3m = 1 \} $$

and prove $p$ or $\lnot p$ is true. Do some examples.

\end{problem}

\begin{solution} The negation of $p$ is

	$$ \lnot p \equiv \{ \exists n \in \mathbb{Z}, \forall m \in \mathbb{Z} \text{ such that } 4n + 3m \neq 0 \text{ and } 4n + 3m \neq 1 \}. $$


\end{solution}


\begin{proof} We will now prove $\lnot p.$ Set $n_0 = 2.$ Then $4n_0 = 8.$ Let $m_0$ be an arbitrary integer. We will consider cases.

	\noindent Case 1. Suppose $m_0 \geq -2.$ Multiplying by 3, we get $3m_0 \geq -6.$ Adding 8 to both sides, we get

	$$8 + 3m_0 \geq 2.$$

	Substituting $4n_0$ for 8, we get

	$$4n_0 + 3m_0 \geq 2.$$


	\noindent Case 2. Suppose $m_0 \leq -3$. Multiplying both sides by 3, we get $3m_0 \leq -9$. Adding 8 to both sides, we get

	$$8 + 3m_0 \leq -1.$$

	But since $4n_0 = 8$, we have

	$$4n_0 + 3m_0 \leq -1$$.

	Since $4n_0 + 3m_0 \neq 0$ and $4n_0 + 3m_0 \neq 1$ for all $m_0$, $\lnot p$ is true.



\end{proof}


\noindent While working on this proof, I received no external assistance aside from advice from Professor Mehmetaj.

\end{document}


% --------------------------------------------------------------
% This is all preamble stuff that you don't have to worry about.
% Head down to where it says "Start here"
% --------------------------------------------------------------

\documentclass[12pt]{article}
 
\usepackage[margin=1in]{geometry} 
\usepackage{amsmath,amsthm,amssymb}
\usepackage{comment}
 

 
\newenvironment{theorem}[2][Theorem]{\begin{trivlist}
\item[\hskip \labelsep {\bfseries #1}\hskip \labelsep {\bfseries #2.}]}{\end{trivlist}}
\newenvironment{lemma}[2][Lemma]{\begin{trivlist}
\item[\hskip \labelsep {\bfseries #1}\hskip \labelsep {\bfseries #2.}]}{\end{trivlist}}
\newenvironment{exercise}[2][Exercise]{\begin{trivlist}
\item[\hskip \labelsep {\bfseries #1}\hskip \labelsep {\bfseries #2.}]}{\end{trivlist}}
\newenvironment{problem}[2][Problem]{\begin{trivlist}
\item[\hskip \labelsep {\bfseries #1}\hskip \labelsep {\bfseries #2.}]}{\end{trivlist}}
\newenvironment{question}[2][Question]{\begin{trivlist}
\item[\hskip \labelsep {\bfseries #1}\hskip \labelsep {\bfseries #2.}]}{\end{trivlist}}
\newenvironment{corollary}[2][Corollary]{\begin{trivlist}
\item[\hskip \labelsep {\bfseries #1}\hskip \labelsep {\bfseries #2.}]}{\end{trivlist}}

% Solutions use a modified proof environment
\newenvironment{solution}
               {\let\oldqedsymbol=\qedsymbol
                \renewcommand{\qedsymbol}{$\blacktriangleleft$}
                \begin{proof}[\textit\upshape Solution]}
               {\end{proof}
                \renewcommand{\qedsymbol}{\oldqedsymbol}}
 
\begin{document}

% --------------------------------------------------------------
%                         Start here
% --------------------------------------------------------------

\title{Individual 7}%replace X with the appropriate number
\author{Yu Fan Mei\\ %replace with your name
	Introduction to Proof and Problem Solving} %if necessary, replace with your course title

\maketitle

\begin{problem}{1}
    Use induction to prove that $(n^3 - n)(n + 2)$ is divisible by $12$ for all $n \geq 1$.
\end{problem}

Before we prove this, we will need a lemma.

\begin{lemma}{1} For every integer greater than 0, $n^3 + 3n^2 + 2n$ is divisible by 3.
\end{lemma}

\begin{proof}
    We will prove this lemma via induction. \\ \\ 
    Base case: Set $n_0 = 1$. Then 
 $n_0^3 + 3n_0^2 + 2n_0 = 6 = 3(2).$ \\ \\
 \noindent
    Inductive Step: Assume there exists an integer $k_0$ and an integer $n_0 \geq 1$ such that

    $$n_0^3 + 3n_0^2 + 2n_0 = 3k_0.$$

    We want to show that there exists an integer $j$ such that

    $$(n_0+1)^3 + 3(n_0+1)^2 + 2(n_0+1) = 3j.$$

    Set $j = j_0 = k_0 + 3n_0^2 + 3n_0 + 2$. Then we know that

    \begin{align*}
        (n_0+1)^3 + 3(n_0+1)^2 + 2(n_0+1) &= n_0^3 + 3n_0^2 + 3n_0 + 1 + 3(n_0^2 + 2n_0 + 1) + 2n_0 + 2\\
        &= n_0^3 + 6n_0^2 + 11n_0 + 6.
    \end{align*}

    We can substitute in $3k_0$, and we get

    \begin{align*}
        (n_0+1)^3 + 3(n_0+1)^2 + 2(n_0+1) &= 3k_0 + 3n_0^2 + 9n_0 + 6 \\
        &= 3(k_0 + n_0^2 + 3n_0 + 2) \\
        &= 3j_0.
    \end{align*}

    Thus, we have proven what we needed to prove.
\end{proof}

\begin{proof} We will prove problem 1 using induction. \\ \\
    Base Case: Set $n = n_0 = 1$. Then
    $$(n_0^3 - n_0)(n_0 + 2) = 0 = 12(0).$$

    \noindent Inductive Step: Assume there exists an integer $n \geq 1$ and $k_0$ such that

    $$(n^3 - n)(n + 2) = 12k_0.$$

    We want to show that there exists an integer $j$ such that

    $$((n_0+1)^3 - (n_0+1))((n_0+1) + 2) = 12j.$$
    
    We need to prove that this is true for $n_0 + 1$, meaning we want to show that there exists an integer $j_0$ such that
    $$((n_0+1)^3 - (n_0+1))((n_0+1) + 2) = 12j_0.$$

    Set $j = j_0 = k_0 + m$, where $m$ is an integer. Unfactoring the left hand side, we get 

    $$((n_0+1)^3 - (n_0+1))((n_0+1) + 2) = n_0^4 + 6n_0^3 + 11n_0^2 + 6n_0.$$

    When we substitute in $12k_0$, we get

    \begin{align*}
        ((n_0+1)^3 - (n_0+1))((n_0+1) + 2) & = 12k_0 + 4n_0^3 + 12n_0^2 + 8n_0 \\
        & = 12k_0 + 4(n_0^3 + 3n_0^2 + 2n_0).
    \end{align*}

    By lemma 1, we know that $n_0^3 + 3n_0^2 + 2n_0$ is divisible by 3. We can rewrite the statement like this:

    \begin{align*}
        ((n_0+1)^3 - (n_0+1))((n_0+1) + 2) & = 12k_0 + 4(3m) \\
        & = 12k_0 + 12m \\
        & = 12(k_0 + m) \\
        & = 12j_0
    \end{align*}

    Thus, we have proven what we needed to prove.


\end{proof}




% problem #2

\newpage
\begin{problem}{2}
    Let r represent an arbitrary real number other than 0 and 1. Show that for $n \in \mathbb{Z}_0^+$

    $$\sum_{i=0}^{n} r^i = \frac{1 - r^{n+1}}{1 - r}.$$
    This is the formula for what is called the finite geometric series. This formula is quite important in many different fields of mathematics and should be committed to memory.

\end{problem}

\begin{proof} We will prove this using mathematical induction. \\ \\

    Base Case: Set $n_0 = 0.$ Then $$\sum_{i=0}^{n_0} r^i = r^0 = 1 = \frac{1 - r^{n_0+1}}{1 - r}.$$

    This statement is clearly true.

    Inductive Step: Assume there exists a nonnegative integer $n_0$ such that for every real number $r$ aside from 0 and 1,

    $$\sum_{i=0}^{n_0} r^i = \frac{1 - r^{n_0+1}}{1 - r}.$$

    We want to show that this is true for $n_0+1$, so we need to show that

    $$\sum_{i=0}^{n_0} r^i + r^{n_0+1} = \sum_{i=0}^{n_0 + 1} r^i.$$

    By our inductive hypothesis, we can rewrite the left side of the equation as

    $$\frac{1 - r^{n_0+1}}{1 - r} + r^{n_0+1} = \sum_{i=0}^{n_0 + 1} r^i.$$

    We can multiply $r^{n_0+1}$ by $(1-r)/(1-r)$ to get a common denominator:

    \begin{align*}
        \sum_{i=0}^{n_0 + 1} r^i & = \frac{1 - r^{n_0+1} + r^{n_0+1}(1-r)}{1 - r} \\
        & = \frac{1 - r^{n_0+1} + r^{n_0+1} - r^{n_0+2}}{1 - r} \\
        & = \frac{1 - r^{n_0+2}}{1 - r}. \\
    \end{align*}

    Thus, we have proven this is true.

\end{proof}


\noindent While working on this proof, I received no external assistance aside from advice from Professor Mehmetaj.

\end{document}
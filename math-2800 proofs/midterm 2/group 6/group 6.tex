% --------------------------------------------------------------
% This is all preamble stuff that you don't have to worry about.
% Head down to where it says "Start here"
% --------------------------------------------------------------
\documentclass[12pt]{article}
\usepackage[margin=1in]{geometry}
\usepackage{amsmath,amsthm,amssymb}
\newenvironment{theorem}[2][Theorem]{\begin{trivlist}
\item[\hskip \labelsep {\bfseries #1}\hskip \labelsep {\bfseries #2.}]}{\end{trivlist}}
\newenvironment{lemma}[2][Lemma]{\begin{trivlist}
\item[\hskip \labelsep {\bfseries #1}\hskip \labelsep {\bfseries #2.}]}{\end{trivlist}}
\newenvironment{exercise}[2][Exercise]{\begin{trivlist}
\item[\hskip \labelsep {\bfseries #1}\hskip \labelsep {\bfseries #2.}]}{\end{trivlist}}
\newenvironment{problem}[2][Problem]{\begin{trivlist}
\item[\hskip \labelsep {\bfseries #1}\hskip \labelsep {\bfseries #2.}]}{\end{trivlist}}
\newenvironment{question}[2][Question]{\begin{trivlist}
\item[\hskip \labelsep {\bfseries #1}\hskip \labelsep {\bfseries #2.}]}{\end{trivlist}}
\newenvironment{corollary}[2][Corollary]{\begin{trivlist}
\item[\hskip \labelsep {\bfseries #1}\hskip \labelsep {\bfseries #2.}]}{\end{trivlist}}
% Solutions use a modified proof environment
\newenvironment{solution}
{\let\oldqedsymbol=\qedsymbol
\renewcommand{\qedsymbol}{$\blacktriangleleft$}
\begin{proof}[\textit\upshape Solution]}
{\end{proof}
\renewcommand{\qedsymbol}{\oldqedsymbol}}
\begin{document}
% --------------------------------------------------------------
% Start here
% --------------------------------------------------------------
\title{Group 6}%replace X with the appropriate number
\author{Axel Abrica, Yu Fan Mei, Abby Porter\\ %replace with your name
Introduction to Proof and Problem Solving} %if necessary, replace with your course title
\maketitle
%--------------------------------------------------
%                 Writing Problem
%--------------------------------------------------

\begin{problem}{1}
    For what values of $n\in\mathbb{O}^+$ is $$\sum^n_{j=0}2^j$$ prime?
\end{problem}
%--------------------------------------------------
%                 Writing Proof
%--------------------------------------------------
\begin{proof}
    The only values of $n\in\mathbb{O}^+$ for which the sum is prime is $n=1$. We will prove this by showing that the sum is divisible by $3$ for all $n\in\mathbb{O}^+.$ That is, we will prove for all $n\in\mathbb{O}^+,$ there exists a $k\in\mathbb{Z}$ such that $$\sum^n_{j=0}2^j=3k.$$\\
    \textit{Base Case:} Set $n_0=1$. Then, $$\sum^1_{j=0}2^j=2^0+2^1=1+2=3=3k_0,$$ where $k_0=1.$\\
    \textit{Induction Step:} We assume there exists an odd positive integer $n_0\geq1$ and an integer $k_0$ such that $$\sum^{n_0}_{j=0}2^j=3k_0.$$ We want to show that there exists an integer $l$ such that $$\sum^{n_0+2}_{j=0}2^j=3l.$$ Set $l_0=k_0+2^{n_0+1}.$ Since integers are closed under addition and multiplication, $l_0\in\mathbb{Z}.$ Rewriting the sum, we get 
    \begin{align*}
        \sum^{n_0+2}_{j=0}2^j&=\sum^{n_0}_{j=0}2^j+2^{n_0+1}+2^{n_0+2}\\
        &=3k_0+2^{n_0}\cdot2+2^{n_0}\cdot2^2\\
        &=3k_0+2^{n_0}(2+2^2)\\
        &=3k_0+2^{n_0}(6)\\
        &=3k_0+3(2^{n_0}\cdot2)\\
        &=3(k_0+2^{n_0+1})\\
        &=3l_0.
    \end{align*}
    So, for all $n_0\in\mathbb{O}^+$, the sum is divisible by $3$. However, the sum only equals $3$ when $n_0=1$, and $3$ is the only value divisible by $3$ that is prime. Thus, for all other odd $n_0>1$, the sum is divisible by $3,$ but not prime. 
\end{proof}









%--------------------------------------------------
%                 Writing Problem
%--------------------------------------------------
\begin{problem}{2}
    For each natural number $n$, the $n$th Fibonacci number $f_n$ is given by
$$f_1 = 1, \,\, f_2 =1, \text{ and } \,\, f_{n+2} = f_{n+1} + f_{n} \text{ for all } n \geq 1.$$
Let $\alpha$ be the positive solution and $\beta$ the negative solution to the equation $x^2 = x +
1$. (The values are $\alpha = (1 + \sqrt{5})/2$ and $\beta = (1 - \sqrt{5})/2.)$ Show for all $n \in
\mathbb{N}$ that
\begin{equation*}
f_n = \dfrac{\alpha^n - \beta^n}{\alpha - \beta}.
\end{equation*}

%--------------------------------------------------
%                 Writing Proof
%--------------------------------------------------

\begin{proof}We start by checking the two base cases. Set $n_0 = 1$. Then
    \begin{align*}
        f_1 &=\frac{\alpha^1 - \beta^1}{\alpha-\beta}\\&=\frac{\alpha-\beta}{\alpha-\beta}\\
        &=1.
    \end{align*}This matches $f_1 = 1$ so the base case holds for $n_0 =1$. We now set $n_0 = 2$. Then we get
    \begin{align*}
        f_2&=\frac{\alpha^2-\beta^2}{\alpha-\beta}.
    \end{align*}We know $\alpha$ and $\beta$ are roots of the equation $x^2 = x+1$. Using the identity $\alpha^2 = \alpha +1$ and $\beta^2 = \beta +1$, we get
    \begin{align*}
        \frac{\alpha^2 - \beta^2}{\alpha-\beta}&=\frac{(\alpha +1)-(\beta+1)}{\alpha-\beta}\\
        &= \frac{\alpha-\beta}{\alpha-\beta}\\
        &=1.
    \end{align*}This matches $f_2 =1$, so the base case holds for $n_0 = 2$. We now move on to the induction step. Assume there exists an integer $n_0 \geq 1$ such that the formula holds true for $n_0$ and $n_0 +1$. That is, we assume
    \begin{align*}
        f_{n_0} = \frac{\alpha^{n_0} - \beta^{n_0}}{\alpha-\beta}
    \end{align*}and 
    \begin{align*}
        f_{n_0+1} = \frac{\alpha^{n_0+1} - \beta^{n_0+1}}{\alpha-\beta}.
    \end{align*} We need to show that the formula holds for $n_0+2$. That is, we need to prove
    \begin{align*}
        f_{n_0+2} = \frac{\alpha^{n_0+2} - \beta^{n_0+2}}{\alpha-\beta}.
    \end{align*}Using the recurrence relation $f_{n+2} = f_{n+1} + f_n$, we have that 
    \begin{equation*}
        f_{n_0+2}=f_{n_0} + f_{n_0+1}.
    \end{equation*}From the induction step, we can substitute the expressions for $f_{n_0}$ and $f_{n_0+1}$ into this equation to get that 
    \begin{align*}
    f_{n_0+2}&=f_{n_0} + f_{n_0+1}\\
    &=\frac{\alpha^{n_0} - \beta^{n_0}}{\alpha-\beta} + \frac{\alpha^{n_0+1} - \beta^{n_0+1}}{\alpha-\beta}\\
    &=\frac{\alpha^{n_0} + \alpha^{n_0+1} - (\beta^{n_0}+\beta^{n_0 +1})}{\alpha-\beta}.
    \end{align*}We know $\alpha$ and $\beta$ are roots of the equation $x^2 = x+1$. Using the identity $\alpha^2 = \alpha +1$ and $\beta^2 = \beta +1$, we can express $\alpha^{n_0+2}$ and $\beta^{n_0+2}$ in terms of previous powers. We get that
    \begin{align*}
        \alpha^{n_0+2} &= \alpha^2 \cdot \alpha^{n_0}\\
        &=(\alpha + 1) \cdot \alpha^{n_0}\\
        &= \alpha^{n_0+1} + \alpha^{n_0}.
    \end{align*}Similarly, we get that
    \begin{equation*}
        \beta^{n_0+2} = \beta^{n_0+1} + \beta^{n_0}.
    \end{equation*}Therefore,
    \begin{align*}
        \alpha^{n_0+2} - \beta^{n_0+2} &= (\alpha^{n_0+1} + \alpha^{n_0}) - (\beta^{n_0+1} + \beta^{n_0})\\
        &= \alpha^{n_0+1} - \beta^{n_0+1} + \alpha^{n_0} - \beta^{n_0}.
    \end{align*}We now substitute this back into our expression for $f_{n_0 + 2}$ to get:
    \begin{align*}
        f_{n_0+2} = \frac{\alpha^{n_0+2} - \beta^{n_0+2}}{\alpha - \beta},
    \end{align*}which is what we needed to show. Thus by induction, the formula $f_n$ holds for all $n \in \mathbb{N}$.
\end{proof}

\end{problem}
% --------------------------------------------------------------
% You don't have to mess with anything below this line.
% --------------------------------------------------------------
\end{document}